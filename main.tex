\documentclass{article}
\usepackage{graphicx} % Required for inserting images

\title{School Bus Tracker}
\author{Yasmini Gyawali}
\date{June 2023}

\begin{document}

\maketitle

\section{Introduction}
The purpose of this system is to provide real-time monitoring and tracking of school buses, allowing school administrators and parents to have accurate information about the bus's location and arrival times in order to enhance the safety, efficiency, and accountability of student transportation.

\subsection{Document Conventions}
The following conventions are used in this document.
{TBD}

\subsection{Intended Audience and Reading Suggestions}
This school bus tracking system prototype is designed specifically for use by schools, making it beneficial for school administrators, parents/guardians, and bus drivers.

\subsection{Project Scope}
The purpose of implementing a GPS tracking system for school buses stems from the need to enhance safety, efficiency, and accountability in student transportation. Many students rely on the school bus for transportation, which often causes parents to feel anxious about their arrival at home or school.Hence, a GPS tracking system provides real-time updates and notifications to parents.

\subsection{References}
{TBD}

\section{Overall Description}
\subsection{Product Perspective}
The school bus tracking system utilizes GPS technology to monitor and track the location of school buses in real-time.The product perspective of a school bus tracking system includes the following key elements:
\begin{itemize}
  \item Hardware Platform: GPS device is  installed in each school bus to track their location accurately. These devices communicate with the central software platform to provide real-time updates on bus movements.
  \item The software platform receives data from GPS devices and processes it to generate meaningful information. It typically includes a user-friendly interface for administrators, drivers, and parents to access the necessary information. 
  \item  Data Communication: The school bus tracking system  transmits information between the GPS devices and the software platform. It can utilize cellular networks, satellite communication, or a combination of both, depending on the available infrastructure and system requirements.
  \item  User Roles: The system provides various user roles, each with specific access levels and functionalities. Administrators have complete control over the system, including managing bus routes and assigning drivers. Drivers have access to their assigned bus's location and any other relevant instructions. Parents/guardians can track their children's bus location and receive notifications or alerts.

\end{itemize}
\subsection{Product Features}
\end{document}